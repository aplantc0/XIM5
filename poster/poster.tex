% Gemini theme
% https://github.com/anishathalye/gemini

\documentclass[final]{beamer}

% ====================
% Packages
% ====================

%\usepackage[T1]{fontenc}
%\usepackage{lmodern}

\usepackage{polyglossia}
\setmainlanguage{english}
\setotherlanguages{russian,czech} % \textlang{russian}{ы}
% russian typesetting works when specifying the fonts
\setmainfont{CMU Serif} % only works with lua
\setsansfont{CMU Sans Serif}

%\usepackage[size=custom,width=120,height=72,scale=1.0]{beamerposter}
\usepackage[orientation=portrait,size=a0,scale=1.2]{beamerposter}

\usetheme{gemini}
\usecolortheme{gemini}
\usepackage{graphicx}
\usepackage{booktabs}
\usepackage{tikz}
\usepackage{pgfplots}
\pgfplotsset{compat=1.14}

% ====================
% Lengths
% ====================

% If you have N columns, choose \sepwidth and \colwidth such that
% (N+1)*\sepwidth + N*\colwidth = \paperwidth
%\newlength{\sepwidth}
%\newlength{\colwidth}
%\setlength{\sepwidth}{0.025\paperwidth}
%\setlength{\colwidth}{0.3\paperwidth}

\newlength{\sepwidth}
\newlength{\colwidth}
\setlength{\sepwidth}{0.025\paperwidth}
\setlength{\colwidth}{0.47\paperwidth}

\newcommand{\separatorcolumn}{\begin{column}{\sepwidth}\end{column}}

% ====================
% Title
% ====================

\title{\textit{Pinus sylvestris} L. drought stress reaction thresholds are captured by both intra- and inter-annual variation in xylem morphology}

\author{Sergei Mikhailov \inst{1-3} \and Marek Fajstavr \inst{1,2} \and Petr Horáček \inst{1,2}}

\institute[MendelU]{\inst{1} Department of Xylogenesis and Biomass Allocation, CzechGlobe, Brno, Czech Republic \samelineand \inst{2} Department of Wood Science and Technology, Mendel University in Brno, Brno, Czech Republic \samelineand \inst{3} Laboratory of Ecology of Plant Communities, Komarov Botanical Institute of the Russian Academy of Sciences, Saint Petersburg, Russian Federation}

% ====================
% Footer (optional)
% ====================

\footercontent{
  \includegraphics[height=5cm]{pics/qr} \hfill
  \hspace{12cm} XIM5 2022, Würzburg \hfill
  %\href{https://www.github.com/aplantc0}{github.com/aplantc0}
  \includegraphics[height=5cm]{pics/logo_mendelu}
  \includegraphics[height=5cm]{pics/logo_czechglobe}
  %\href{mailto:mikhailov.s@czechglobe.cz}{mikhailov.s@czechglobe.cz}
  % (can be left out to remove footer)
}

% ====================
% Logo (optional)
% ====================

% use this to include logos on the left and/or right side of the header:
%\logoleft{\includegraphics[height=7cm]{pics/logo_mendelu}}
%\logoright{\includegraphics[height=5cm]{pics/logo_czechglobe}}

% ====================
% Body
% ====================

\begin{document}

\begin{frame}[t]
\begin{columns}[t]
\separatorcolumn

\begin{column}{\colwidth}

    \begin{alertblock}{Context}
        \begin{itemize}
            \item \textbf{Wood formation dynamics} in \textbf{Scots pine} is still not explained partly due to the more complex wood anatomy.
            \item Biggest xylem cells in angiosperm are the \textbf{vessel} cells.
            \item The \textbf{gradient of size and frequency} of vessel over the tree-rings one of the characteristic of the earlywood-latewood transition in angiosperms.
            \item Recent studies bring new insights on the impact of climatic parameters on vessel development but the dynamic of formation is still unknown.
            \item \textbf{Water regime} is the main factor impacting \textbf{cell development} through its action on the turgor pressure.
            \item \textbf{Water use} by the tree could be evaluate via the \textbf{sap flow} density trait.
        \end{itemize}
    \end{alertblock}

    \begin{block}{Objectives}
        \begin{enumerate}
            \item To evaluate the xylem vessel formation dynamic in ring- and diffuse- porous species;
            \item To explore the relationship between the timing of vessel developmental phases and seasonal sap flow pattern by taking account of climatic conditions and the water use strategy of the tree species.
        \end{enumerate}
    \end{block}

    \begin{block}{Water use: Sap flow under water stress episodes}
        askjdasdj
    \end{block}

    \begin{block}{Ring- vs. diffuse- porous wood structure}
        ajsdnajksdnkaj
    \end{block}

\end{column}

\separatorcolumn

% ===================

\begin{column}{\colwidth}

    \begin{block}{Approach}
        \begin{itemize}
            \item Stem radial growth and sap flow density followed over the years 2017, 2018 and 2019. 
            \item Estimation of Tree Water Deficit episodes following Zweiffel et al. (2016).
            \item Wood formation has been monitoring by sampling wood microcores weekly from April to November.
Counting of all xylem cells in each developmental phases and measurement of the vessel area on all samples. 
                Vessels in secondary cell wall deposition are pooled with mature vessels.
        \end{itemize}
    \end{block}

    \begin{block}{Sap flow and wood formation dynamics}
        aksdjakjsdaksjd
    \end{block}

	\end{column}

\end{columns}

\begin{columns}[c]
    \begin{column}{.96\paperwidth}

    \begin{block}{What is next?}
        \raggedleft
        \begin{itemize}
            \item To include growth and wood formation data from 2017 and 2019, and oak wood formation dynamic.
            \item To estimate a time residence in enlargement phase and total developmental duration for vessel formation.
            \item To explore the temporal correlation between sap flow and vessel timing in order to estimate when vessels are functional.
        \end{itemize}
    \end{block}
\end{column}

\separatorcolumn

\end{columns}
\end{frame}

\end{document}
